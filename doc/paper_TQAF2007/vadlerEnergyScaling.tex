\documentclass{cmspaper}

\newcommand{\SnS}{Energy Shifting{\&}Smearing Module}
\newcommand{\Sns}{Energy shifting{\&}smearing module}
\newcommand{\sns}{energy shifting{\&}smearing module}


\begin{document}

 \section{TQAF Layer-1: Reconstruction and selection of analysis-level physics objects}

  \subsection{TQAF Layer-1: Object specific tools} \label{sec:layer1tools}

   \subsubsection{Object energy shifting and smearing} \label{TOOLSSNS}

    The \sns\ provides the possibilty to check systematic errors from the energy scale uncertainties quickly without the need to re-run the whole analysis chain. In this section, the status of this module as found in the TQAF versions for the 2007 analyses is described.

    \begin{description}

     \item[Implementation]
     
      The \sns\ code is implemented as a class template in file\\
      {\tt TopQuarkAnalysis/TopObjectProducers/interface/TopObjectEnergyScale.h},\\
      which takes the type of the object to scale as argument.
      For the use within the TQAF, a concrete instance of the module is foreseen for the types {\tt TopElectron}, {\tt TopMuon}, {\tt TopTau}, {\tt TopJet} and {\tt TopMET}, in classes called {\tt TopElectronEnergyScale} etc.
      However, since only very basic methods of the reconstructed objects (inherited from the class {\tt reco::Particle}) are called, the module is usable with any object inheriting from that class.

      The energy shifting is realized by means of a configurable factor, which is multiplied to the object's energy fourvector.
      In case of a non-positive factor, no action is taken.
      The energy smearing is performed with a Gaussian, whose width is calculated from the configurable initial and final resolution with
      \begin{equation}
       \label{EQNSIGMAGAUSS}
       \sigma_{\rm Gauss} = \sqrt{\sigma_{\rm final}^2-\sigma_{\rm ini}^2} \mbox{,}
      \end{equation}
      where $\sigma_{\rm Gauss}$ is the width of the Gaussian and $\sigma_{\rm ini,final}$ are respectively the initial and final resolution, all given in units of the object's energy. The resulting random number is multiplied to the object's energy fourvector. If $\sigma_{\rm final}$ is not greater than $\sigma_{\rm ini}$ or the object's energy is negative, no action is taken. The smearing does not allow the object's energy to become negative, but rather sets it to $0\,$GeV.

     \item[Usage]
     
      In the file {\tt TopQuarkAnalysis/TopObjectProducers/data/TopObjectEnergyScale.cfi}, the \sns\ provides four configurable parameters, which are explained in Tab. \ref{TABSNSCONFIGS}.
      \begin{table}[width=\textwidth]
       \begin{center}
        \begin{tabular}{*{2}{|l}|c|p{8cm}|}
         \hline
         Parameter               & Type           & \multicolumn{1}{|l|}{Default} & Description\\
         \hline\hline
         {\tt scaledTopObject}   & {\tt InputTag} & {\it none}                    & Defines the input object to scale.
                                                                                    It should be of the same type the module is instanciated for, e.g. {\tt selectedLayer1TopElectrons} of type {\tt TopElectron} in {\tt TopElectronEnergyScale}.\\
         \hline
         {\tt shiftFactor}       & {\tt double}   & $1.0$                         & Factor to be multiplied to the object's fourvector.\\
         \hline
         {\tt initialResolution} & {\tt double}   & $0.0$                         & Initial resolution in units of the object's energy to be used as $\sigma_{\rm ini}$ in (\ref{EQNSIGMAGAUSS}).\\
         \hline
         {\tt finalResolution}   & {\tt double}   & $0.0$                         & Final resolution in units of the object's energy to be used as $\sigma_{\rm final}$ in (\ref{EQNSIGMAGAUSS}).
                                                                                    If this value is smaller or equal to {\tt initialResolution} (as for the default values), the smearing functionality is switched off.\\
         \hline
        \end{tabular}
        \caption{Configurable parameters of the \sns .}
        \label{TABSNSCONFIGS}
       \end{center}
      \end{table}

     \item[Examples]
     
      Only configurable parameters of non-default values are shown.
      \begin{itemize}
       \item Add smeared jets to the event:\\
        \verb!module scaledTopJets = TopJetEnergyScale {!\\
        \verb!  InputTag scaledTopObject   = selectedLayer1TopJets!\\
        \verb!  double   initialResolution = 1.!\\
        \verb!  double   finalResolution   = 2. ...}!\\
        The smearing of the jet energies spoils the pre-selection cut on the TopJets' energy.
       \item Shift muon energies:\\
        \verb!module scaledTopMuons = TopMuonEnergyScale {!\\
        \verb!  InputTag scaledTopObject = selectedLayer1TopMuons!\\
        \verb!  double   shiftFactor     = 1.25 ...}!\\
        The energy shift is visible in the mean value of the TopMuons' energy distribution.
      \end{itemize}
      Histograms illustrating these examples can be found in \cite{Adler:2007a}. Further examples are available on \cite{Adler:2007c}.
      An example on how to incorporate the \sns\ into the analysis chain is given in the file {\tt TopQuarkAnalysis/Examples/test/TtSemiLepEventsReco\_fromLayer1Objects\_scaledObjects.cfg}, which includes {\tt TopQuarkAnalysis/TopEventProducers/data/TQAFLayer2\_TtSemi\_scaledObjects.cff}.

    \end{description}
    

\section{Outlook for the future - ongoing developments} \label{sec:future}

 \subsection{\SnS } \label{FUTSNS}

  Beyond the CVS branches referred to in \ref{TOOLSSNS}, further developments have been implemented in the CVS head \cite{Adler:2007b,Adler:2007c}.

  In the \sns , the following features have been added:
  \begin{itemize}
   \item
    The parameter {\tt finalResolution} is substituted by {\tt worsenResolution}, which corresponds to $\sigma_{Gauss}$ in eqn. \ref{EQNSIGMAGAUSS}.
   \item
    The initial resolution can be given either in units of the object's energy or as absolute value in $GeV$.
   \item
    The worsening resolution can be given either in units of the initial resolution or as absolute value in $GeV$.
   \item
    The energy resolution stored in TopObjects can optionally be used as initial resolution.
    Of course, this limits the usage of the module to TopObjects.
  \end{itemize}
  Also an option to fix the object's mass is foreseen, but suffers currently from inconsistencies in the mass implementation of the input fourvector type.

  In addition, a new module for systematic checks on the spatial resolution (angular smearing) is implemented\cite{Adler:2007d}.
  Here, the angles resp. the pseudo-rapidity of the fourvector are smeared with a Gaussian.
  The configurable parameters correspond to those of the \sns .
  TopObject's initial resolutions are also available.


\begin{thebibliography}{mm}
 \bibitem{Adler:2007a}
  V.~Adler,
  {\em Energy scale shifting and smearing in TQAF},\\
  {\tt http://indico.cern.ch/conferenceDisplay.py?confId=22614}.
 \bibitem{Adler:2007b}
  V.~Adler,
  {\em Tools for scale and resolution systematics},\\
  {\tt http://indico.cern.ch/conferenceDisplay.py?confId=23467}.
 \bibitem{Adler:2007c}
  {\tt https://twiki.cern.ch/twiki/bin/view/CMS/TWikiTQAFEnergyScaling}.
 \bibitem{Adler:2007d}
  {\tt https://twiki.cern.ch/twiki/bin/view/CMS/TWikiTQAFAngularSmearing}.
\end{thebibliography}


\end{document}
